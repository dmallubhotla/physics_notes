\documentclass[../../main.tex]{subfiles}
\newcommand{\vf}{v_F}
\allowdisplaybreaks

\begin{document}

% !TeX spellcheck = en_GB
\section{Explicit parts of Lindhard function}

We want to find the explicit real and imaginary parts of the Lindhard function. To begin with, we can start with the case where $\nu \rightarrow 0$, which means long relaxation times.

We have our Lindhard form
\begin{equation} \label{eq:LindhardExplicitParts:LindhardEps} 
 \epsilon_l = 1 + \frac{3 \omega_p^2}{k^2 v_F^2} \frac{\left(\omega + i \nu \right) f_l(\flatfrac{(\omega + i \nu)}{k v_f})}{\omega + i \nu f_l(\flatfrac{(\omega + i \nu)}{k v_f})},
\end{equation}
where 
\begin{equation}
	f_l(x) = 1 - \frac{x}{2} \ln\frac{x + 1}{x - 1}.
\end{equation}

\subsection{Pines result}

From Pines, we have the forms

\begin{align} 
	\Re[\epsilon_l]	= 1 &+ \frac{k_{TF}^2}{k^2} \Biggl(	
		\frac12 + \frac{k_F}{4k} \biggl[ 
			\left( 1 - \frac{\left(\omega - \frac{\hbar k^2}{2m} \right)^2}{k^2\vf^2} \right) \ln\left[\frac{\omega - k \vf - \frac{\hbar k^2}{2m}}{\omega + k \vf - \frac{\hbar k^2}{2m}}\right] \nonumber \\
			&+ \left(1 - \frac{\left(\omega + \frac{\hbar k^2}{2m} \right)^2}{k^2 \vf^2} \right) \ln\left[\frac{\omega + k\vf + \frac{\hbar k^2}{2m}}{\omega - k \vf + \frac{\hbar k^2}{2m}}\right]			
		\biggr]
	 \Biggr)  \label{eq:LindhardExplicitParts:PinesReal}
\end{align}

\begin{equation}
	\Im[\epsilon_l] = \begin{cases}
		\dfrac{\pi}{2} \dfrac{\omega}{k\vf} \dfrac{k_{TF}^2}{k^2}, & \omega \le k \vf - \frac{\hbar k^2}{2m} \\
		\dfrac{\pi}{4} \dfrac{k_F}{k} \left( 1 - \dfrac{\left(\omega - \frac{\hbar k^2}{2m}\right)^2 }{k^2\vf^2} \right)\dfrac{k_{TF}^2}{k^2}, & k \vf - \frac{\hbar k^2}{2m} \le \omega \le k \vf + \frac{\hbar k^2}{2m} \\ 
		0, & \omega \ge k \vf + \frac{\hbar k^2}{2m}  \\
	\end{cases} \label{eq:LindhardExplicitParts:PinesImaginary}
\end{equation}


\subsection{Long relaxation time forms of the logs}
In order to analyse the $\nu \rightarrow 0$ limit, we can start by looking at what happens to the logarithms in the Lindhard function:
\begin{equation}
	\ln\frac{\omega + i \nu + k\vf}{\omega + i \nu - k\vf}
\end{equation}
The first thing to note is that the numerator will always have a very small, positive argument, while the denominator will have a small argument which may be positive or negative. As Lindhard mentions, these logarithms should all have imaginary parts between $\pm i \pi$. This effectively means we can treat each logarithm as giving the principal value, which give us a result that looks like
\begin{equation}
	\ln\frac{\sqrt{(\omega + k\vf) ^2 + \nu^2}}{\sqrt{(\omega - k\vf) ^2 + \nu^2}} + i \left(\theta_+ - \theta_- \right),
\end{equation}
where $\theta_+$ and $\theta_-$ are the arguments of the numerator and denominator. For small $\nu$, $\theta_+$ is proportional to $\nu$, as it'll be determined by an arcsine. However, the denominator may be negative, which would contribute a factor of $\theta_- = +i\pi$ (with a plus sign because $\nu$ would be just above the real line).
\begin{align}
	\ln\frac{\omega + i \nu + k\vf}{\omega + i \nu - k\vf} &= \ln\frac{\sqrt{(\omega + k\vf) ^2 + \nu^2}}{\sqrt{(\omega - k\vf) ^2 + \nu^2}} + i \left(\theta_+ - \theta_- \right) \\
	&= \ln\frac{\sqrt{(\omega + k\vf) ^2}}{\sqrt{(\omega - k\vf) ^2}} - \sigma i \pi \\
	&= \ln\abs{\frac{\omega + k\vf}{\omega - k\vf}} - \sigma i \pi,
\end{align}
where $\sigma = 1$ if $\omega < k\vf$, and $0$ otherwise.

\end{document}
