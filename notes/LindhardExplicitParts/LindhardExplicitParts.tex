\documentclass[../../main.tex]{subfiles}
\newcommand{\vf}{v_F}
\allowdisplaybreaks

\begin{document}

% !TeX spellcheck = en_GB
\section{Explicit parts of Lindhard function}

We want to find the explicit real and imaginary parts of the Lindhard function. To begin with, we can start with the case where $\nu \rightarrow 0$, which means long relaxation times.

We have our Lindhard form
\begin{equation} \label{eq:LindhardExplicitParts:LindhardEps} 
 \epsilon_l = 1 + \frac{3 \omega_p^2}{k^2 v_F^2} \frac{\left(\omega + i \nu \right) f_l(\flatfrac{(\omega + i \nu)}{k v_f})}{\omega + i \nu f_l(\flatfrac{(\omega + i \nu)}{k v_f})},
\end{equation}
where 
\begin{equation}
	f_l(x) = 1 - \frac{x}{2} \ln\frac{x + 1}{x - 1}.
\end{equation}

\subsection{Pines result}

From Pines, we have the forms

\begin{align} 
	\Re[\epsilon_l]	= 1 &+ \frac{k_{TF}^2}{k^2} \Biggl(	
		\frac12 + \frac{k_F}{4k} \biggl[ 
			\left( 1 - \frac{\left(\omega - \frac{\hbar k^2}{2m} \right)^2}{k^2\vf^2} \right) \ln\left[\frac{\omega - k \vf - \frac{\hbar k^2}{2m}}{\omega + k \vf - \frac{\hbar k^2}{2m}}\right] \nonumber \\
			&+ \left(1 - \frac{\left(\omega + \frac{\hbar k^2}{2m} \right)^2}{k^2 \vf^2} \right) \ln\left[\frac{\omega + k\vf + \frac{\hbar k^2}{2m}}{\omega - k \vf + \frac{\hbar k^2}{2m}}\right]			
		\biggr]
	 \Biggr)  \label{eq:LindhardExplicitParts:PinesReal}
\end{align}

\begin{equation}
	\Im[\epsilon_l] = \begin{cases}
		\dfrac{\pi}{2} \dfrac{\omega}{k\vf} \dfrac{k_{TF}^2}{k^2}, & \omega \le k \vf - \frac{\hbar k^2}{2m} \\
		\dfrac{\pi}{4} \dfrac{k_F}{k} \left( 1 - \dfrac{\left(\omega - \frac{\hbar k^2}{2m}\right)^2 }{k^2\vf^2} \right)\dfrac{k_{TF}^2}{k^2}, & k \vf - \frac{\hbar k^2}{2m} \le \omega \le k \vf + \frac{\hbar k^2}{2m} \\ 
		0, & \omega \ge k \vf + \frac{\hbar k^2}{2m}  \\
	\end{cases} \label{eq:LindhardExplicitParts:PinesImaginary}
\end{equation}


\subsection{Long relaxation time forms of the logs}
In order to analyse the $\nu \rightarrow 0$ limit, we can start by looking at what happens to the logarithms in the Lindhard function:
\begin{equation}
	\ln\frac{\omega + i \nu + k\vf}{\omega + i \nu - k\vf}
\end{equation}
The first thing to note is that the numerator will always have a very small, positive argument, while the denominator will have a small argument which may be positive or negative. As Lindhard mentions, these logarithms should all have imaginary parts between $\pm i \pi$. This effectively means we can treat each logarithm as giving the principal value, which give us a result that looks like
\begin{equation}
	\ln\frac{\sqrt{(\omega + k\vf) ^2 + \nu^2}}{\sqrt{(\omega - k\vf) ^2 + \nu^2}} + i \left(\theta_+ - \theta_- \right),
\end{equation}
where $\theta_+$ and $\theta_-$ are the arguments of the numerator and denominator. For small $\nu$, $\theta_+$ is proportional to $\nu$, as it'll be determined by an arcsine. However, the denominator may be negative, which would contribute a factor of $\theta_- = +i\pi$ (with a plus sign because $\nu$ would be just above the real line).
\begin{align}
	\ln\frac{\omega + i \nu + k\vf}{\omega + i \nu - k\vf} &= \ln\frac{\sqrt{(\omega + k\vf) ^2 + \nu^2}}{\sqrt{(\omega - k\vf) ^2 + \nu^2}} + i \left(\theta_+ - \theta_- \right) \\
	&= \ln\frac{\sqrt{(\omega + k\vf) ^2}}{\sqrt{(\omega - k\vf) ^2}} - \sigma i \pi \\
	&= \ln\abs{\frac{\omega + k\vf}{\omega - k\vf}} - \sigma i \pi,
\end{align}
where $\sigma = 1$ if $\omega < k\vf$, and $0$ otherwise.

\subsection{Taking long scattering time limit}

To make our constants line up with the Pines form, we'll use the definition $k_{TF}^2 = \frac{3 \omega_p^2}{\vf^2}$, giving us 
\begin{align} 
 \epsilon_l &= 1 + \frac{k_{TF}^2}{k^2} \frac{\left(\omega + i \nu \right) f_l(\flatfrac{(\omega + i \nu)}{k v_f})}{\omega + i \nu f_l(\flatfrac{(\omega + i \nu)}{k v_f})},
\end{align}
We can't simply set $\nu = 0$, as we need to respect the relation
\begin{equation}
	\frac{1}{x + i \delta} = \frac{1}{x} - i \pi \delta(x)	
\end{equation}

\begin{align} 
 \epsilon_l &= 1 + \frac{k_{TF}^2}{k^2} \frac{\left(\omega + i \nu \right) \left( 1 - \frac{\omega + i \nu}{2 k\vf} \left(\ln\abs{\frac{\omega + k\vf}{\omega - k\vf}} - \sigma i \pi\right)\right)}{\omega + i \nu\left( 1 - \frac{\omega + i \nu}{2 k\vf} \left(\ln\abs{\frac{\omega + k\vf}{\omega - k\vf}} - \sigma i \pi\right)\right)}.
\end{align}
We can eliminate all the terms proportional to $\nu$ in the numerator, as they will disappear as we take our limit:
\begin{align}
	\epsilon_l &= 1 + \frac{k_{TF}^2}{k^2} \frac{\left(\omega + i \nu \right) \left( 1 - \frac{\omega + i \nu}{2 k\vf} \left(\ln\abs{\frac{\omega + k\vf}{\omega - k\vf}} - \sigma i \pi\right)\right)}{\omega + i \nu\left( 1 - \frac{\omega + i \nu}{2 k\vf} \left(\ln\abs{\frac{\omega + k\vf}{\omega - k\vf}} - \sigma i \pi\right)\right)} \\
	\epsilon_l &= 1 + \frac{k_{TF}^2}{k^2} \frac{\omega \left( 1 - \frac{\omega}{2 k\vf} \left(\ln\abs{\frac{\omega + k\vf}{\omega - k\vf}} - \sigma i \pi\right)\right)}{\omega + i \nu\left( 1 - \frac{\omega + i \nu}{2 k\vf} \left(\ln\abs{\frac{\omega + k\vf}{\omega - k\vf}} - \sigma i \pi\right)\right)} \\
	\epsilon_l &= 1 + \frac{k_{TF}^2}{k^2} \frac{\omega}{2 k \vf} \frac{ 2 k\vf - \omega \ln\abs{\frac{\omega + k\vf}{\omega - k\vf}} + i \omega \pi \sigma}{\omega + i \nu\left( 1 - \frac{\omega + i \nu}{2 k\vf} \left(\ln\abs{\frac{\omega + k\vf}{\omega - k\vf}} - \sigma i \pi\right)\right)} \label{eq:LindhardExplicitParts:NumeratorSimp}
\end{align}
We can now look at just the denominator:
\begin{align}
	&= \omega + i \nu\left( 1 - \frac{\omega + i \nu}{2 k\vf} \left(\ln\abs{\frac{\omega + k\vf}{\omega - k\vf}} - \sigma i \pi\right)\right) \\
	&= \omega + i \nu - i\nu\frac{\omega + i \nu}{2 k\vf} \left(\ln\abs{\frac{\omega + k\vf}{\omega - k\vf}} - \sigma i \pi\right) \\
	&= \omega + i \nu + \frac{\nu^2 - i \nu \omega}{2 k \vf}  \left(\ln\abs{\frac{\omega + k\vf}{\omega - k\vf}} - \sigma i \pi\right) \\
	&= \omega + \frac{\nu^2}{2 k \vf} L - \frac{\nu \omega \pi \sigma}{2 k \vf} + i\nu - i\frac{\nu \omega}{2 k \vf} L - i \frac{\nu^2 \sigma \pi}{2 k \vf},
\end{align}
where $L = \ln\abs{\frac{\omega + k\vf}{\omega - k\vf}}$. We can notice here that all the imaginary terms are proportional to $\nu$. We can ignore the $\nu^2$ term, as it will go to zero faster than the other terms. We can see that we essentially have two regimes: if $\omega < 2 k\vf$, this will have a positive imaginary part, and if $\omega > 2k\vf$, the imaginary part will be negative.

We can also notice that the only real part that will survive the limiting process is simply $\omega$ (which is of course clear from the original form of the denominator anyway). This lets us essentially write the denominator as 
\begin{align}
	\omega + i \nu \zeta C,
\end{align}
where I'm defining $\zeta$ as $1$ if $\omega < 2k\vf$, and $-1$ otherwise. We also can note that $C$ is an irrelevant positive constant; when we take the long scattering time limit, all that matters is that the imaginary part is proportional to $\nu$. Plugging this back into \eqref{eq:LindhardExplicitParts:NumeratorSimp} gives us 
\begin{align}
	\epsilon_l &= 1 + \frac{k_{TF}^2}{k^2} \frac{\omega}{2 k \vf} \frac{ 2 k\vf - \omega \ln\abs{\frac{\omega + k\vf}{\omega - k\vf}} + i \omega \pi \sigma}{\omega + i \nu\left( 1 - \frac{\omega + i \nu}{2 k\vf} \left(\ln\abs{\frac{\omega + k\vf}{\omega - k\vf}} - \sigma i \pi\right)\right)} \\
	\epsilon_l &=  1 + \frac{k_{TF}^2}{k^2} \frac{\omega}{2 k \vf} \frac{ 2 k\vf - \omega \ln\abs{\frac{\omega + k\vf}{\omega - k\vf}} + i \omega\pi \sigma}{\omega + i \nu \zeta C}  \\
	&= 1 + \frac{k_{TF}^2}{k^2} \frac{\omega}{2 k \vf} \left( 2 k\vf - \omega \ln\abs{\frac{\omega + k\vf}{\omega - k\vf}} + i \omega \pi \sigma\right)\left(\frac{1}{\omega} - i \pi \zeta \delta(\omega) \right) \\
	&= 1 + \frac{k_{TF}^2}{k^2} \frac{\omega}{2 k \vf} \left( \frac{2k\vf}{\omega}- L + i \pi \sigma - i \pi \zeta 2 k \vf \delta(\omega)  +\left( i  L \zeta + \sigma \zeta\right) \pi \omega \delta(\omega)\right)
\end{align}
We can eliminate the terms proportional to $\omega \delta(\omega)$:
\begin{align}
	\epsilon_l 	&= 1 + \frac{k_{TF}^2}{k^2} \frac{\omega}{2 k \vf} \left( \frac{2k\vf}{\omega}- L + i \pi \sigma - i \pi \zeta 2 k \vf \delta(\omega) \right) \\
	&= 1 + \frac{k_{TF}^2}{k^2} \left( 1 -\frac{\omega}{2 k \vf} L + \frac{\omega}{2 k \vf} i\pi \sigma - i \pi \zeta\omega \delta(\omega) \right) \\
	&= 1 + \frac{k_{TF}^2}{k^2} \left( 1 -\frac{\omega}{2 k \vf} L + \frac{\omega}{2 k \vf} i \pi\sigma  \right) 
\end{align}

\end{document}
