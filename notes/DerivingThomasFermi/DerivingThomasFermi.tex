\documentclass[../../main.tex]{subfiles}

\begin{document}

% !TeX spellcheck = en_GB
\section{General discussion of dielectric function}

We can start by looking at some basic relationships between $\vec{D}$, $\vec{E}$. In SI units, and writing $\epsilon$ instead of $\epsilon_r$ to avoid unnecessary subscripts,
\begin{equation}
	\vec{D}_\alpha(r) = \int \dd[d]{r'} \epsilon_{\alpha\beta}(r, r') \epsilon_0 \vec{E}_\beta(r').
\end{equation}

We can start by making the large assumption that the system is isotropic, and thus, the position dependence for $\epsilon$ must be of the form $\epsilon_{\alpha \beta}(r - r')$. This justifies the following:

\begin{align}
	\vec{D}_\alpha(r) &= \int \dd[d]{r'} \epsilon_{\alpha\beta}(r - r') \epsilon_0 \vec{E}_\beta(r')\\
	&= \int \dd[d]{r'} \epsilon_{\alpha\beta}(r - r') \epsilon_0 \vec{E}_\beta(r') e^{i q r'} e^{-i q r'}\\
	&= \int \dd[d]{r'} \epsilon_{\alpha\beta}(r - r') \epsilon_0 \vec{E}_\beta(r') e^{i q r'} e^{-i q r'} \\
	&= \int \dd[d]{r'} \epsilon_{\alpha\beta}(r - r') e^{i q r'} \epsilon_0 \vec{E}_\beta(r')  e^{-i q r'} \\
	\vec{D}_\alpha(r) e^{-i q r} &= \int \dd[d]{r'} \epsilon_{\alpha\beta}(r - r') e^{i q r'}  e^{-i q r} \epsilon_0 \vec{E}_\beta(r')  e^{-i q r'} \\
	\vec{D}_\alpha(r) e^{-i q r} &= \int \dd[d]{r'} \epsilon_{\alpha\beta}(r - r') e^{-iq(r - r')} \epsilon_0 \vec{E}_\beta(r')  e^{-i q r'}
\end{align}
If we integrate this over $r$, and recognise our Fourier transforms, this becomes
\begin{equation}
	D_\alpha(q) = \epsilon_{\alpha \beta}(q) E_\beta(q)
\end{equation}
This is a fair result for the assumption of isotropy. We have an implicit sum over $\beta$ here to look at.

Skipping over some details to fill in later, we end up with
\begin{equation}
	\frac{1}{\epsilon_r(q, \omega)} = 1 + \frac{4\pi e^2}{q^2} \Pi(q, \omega),
\end{equation}
where $\Pi(q, \omega)$ is a response function satisfying
\begin{equation}
	n_{ind}(q, \omega) = \Pi(q, \omega) V_f(q, \omega)
\end{equation}
Here $n_ind$ is the number density of induced electrons, and $V_f$ is the voltage created by any free electrons in the metal (which isn't quite the same as an external voltage, but I think you might be able to ignore that difference).

\end{document}
