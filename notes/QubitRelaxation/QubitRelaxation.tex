\documentclass[../../main.tex]{subfiles}
\newcommand{\vf}{v_F}
\allowdisplaybreaks

\begin{document}

% !TeX spellcheck = en_GB
\section{Qubit Relaxation Time}

\subsection{The quasi-static limit}
We can start by looking at
\begin{equation}
	\chi_{zz}^{E}(z, z, \omega) = \frac{\hbar}{\epsilon_0}\Re\int_0^\infty \dd{p} \frac{p^3}{q} e^{2iqz}r_p(p)
\end{equation}
Here, we have
\begin{equation}
	q = \begin{cases}
			\sqrt{\frac{\omega^2}{c^2} - p^2}, & p^2 \le  \frac{\omega^2}{c^2} \\
			i \sqrt{p^2 - \frac{\omega^2}{c^2}}, & p^2 > \frac{\omega^2}{c^2}
		\end{cases}
\end{equation}
If we look at the case where $\omega = \SI{6\pi e8}{\per\second}$, the cutoff for real or imaginary $q$ will be when $p = \SI{2\pi}{\per\meter}$.

If we assume that $\Im r_p$ doesn't decay too quickly, this integral will be dominated by values of $p$ larger than this, which lets us make the substitution that $\frac{\omega}{c} \rightarrow 0$, which means this integral will reduce to
\begin{align}
	\chi_{zz}^{E}(z, z, \omega) = \frac{\hbar}{\epsilon_0}\int_0^\infty \dd{p} p^2 e^{-2pz}\Im r_p(p, \omega)
\end{align}
The note that we're effectively taking $c\rightarrow \infty$ is important, as we still shouldn't necessarily assume that we can take $\omega \rightarrow 0$ in $r_p$.

\subsection{Non-local reflection coefficient}
We can look specifically at what $r_p$ will be in the non-local case:
\begin{equation}
	r_p(p, \omega) = \frac{1 - \frac{2p}{\pi} \int_0^\infty \dd{\kappa} \frac{1}{k^2 \epsilon_l(k \omega)}}{1 + \frac{2p}{\pi} \int_0^\infty \dd{\kappa} \frac{1}{k^2 \epsilon_l(k \omega)}}
\end{equation}
where $k^2 = p^2 + \kappa^2$ and
\begin{equation}
	\epsilon_l = 1 + \frac{3 \omega_p^2}{k^2 v_F^2} \frac{\left(\omega + i \nu \right) f_l(\flatfrac{(\omega + i \nu)}{k v_f})}{\omega + i \nu f_l(\flatfrac{(\omega + i \nu)}{k v_f})}.
\end{equation}

All of the interesting behaviour here comes from the integral, which we might name $I = \int_0^\infty \dd{\kappa} \frac{1}{k^2 \epsilon_l(k,m \omega)}$. Knowing that we will eventually need to find $\Im r_p$, we might find utility in also writing $I = I_1 + i I_2$ and noting that
\begin{align}
	\Im r_p &= \Im \frac{1 - I}{1 + I} \\
	&= \Im \frac{1 - I_1 - i I_2}{1 + I_1 + i I_2} \\
	&= \Im \frac{1 - I_1 - i I_2}{1 + I_1 + i I_2} \frac{I + I_1 - i I_2}{I + I_1 - i I_2} \\
	&= \Im \frac{(1 - I_1)(1 + I_1) - I_2^2 - i I_2(1 - I_1 + 1 + I_1)}{(1 + I_1)^2 + I_2^2} \\
	&= \Im \frac{(1 - I_1)(1 + I_1) - I_2^2 - 2 i I_2}{(1 + I_1)^2 + I_2^2} \\
	&= \frac{- 2 I_2}{(1 + I_1)^2 + I_2^2}.
\end{align}
This gives us some idea of how $\Im r_p$ should behave, at least once we can write out the integral $I$.

\begin{align}
	I &= \int_0^\infty \dd{\kappa} \frac{1}{k^2 \epsilon_l(k, \omega)}
\end{align}

\end{document}
