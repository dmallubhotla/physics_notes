\documentclass[../../main.tex]{subfiles}
\newcommand{\sigmaDC}{\sigma_{\textrm{DC}}}
\newcommand{\sigmaAC}{\sigma_{\textrm{AC}}}


\begin{document}

% !TeX spellcheck = en_GB
\section{Drude model parameters}

The assumptions of the Drude model are simple: we have interaction-free electrons that occasionally undergo some scattering process during a time $\dd{t}$ with probability $\frac{\dd{t}}{\tau}$, where $\tau$ is some phenomenological parameter. This scattering will randomise electron momentum.

Our ultimate goal will be to find the conductivity $\sigma$ and the dielectric constant $\epsilon$ in the Drude model, with Drude relaxation time $\tau$, electron density $n$ and electron mass $m$. We'll find

\begin{align}
	\sigmaDC &= \frac{n e^2 \tau}{m} \\
	\sigmaAC &= \frac{n e^2 \tau}{m} \frac{1}{1 - i \omega \tau} && \text{For SI and Gaussian} \label{eq:DrudeTheory:SigmaAC}
\end{align}

\begin{subequations}
	\begin{align}
		\epsilon_r &= 1 + i \frac{4 \pi \sigma}{\omega} && \text{Gaussian} \\
		\epsilon_r &= 1 + i \frac{\sigma}{\omega \epsilon_0} && \text{SI} 
	\end{align}
\end{subequations}

Our dielectric constant can be rewritten to plug in for $\sigma$, giving us
\begin{align}
	\epsilon &= 1 + i \frac{4 \pi \sigma}{\omega} \\
	&= 1 + i \frac{4 \pi \sigmaDC}{\omega} \frac{1}{1 - i \omega \tau} \\
 &= 1 + i \frac{4 \pi \sigmaDC}{\omega} \frac{1}{1 - i \omega \tau} \frac{1 + i \omega\tau}{1 + i \omega \tau} \\
 &= 1 + i \frac{4 \pi \sigmaDC}{\omega} \frac{1 + i \omega \tau}{1 + \omega^2 \tau^2} \\
 &= \left(1 - \frac{4 \pi \sigmaDC \omega \tau}{\omega\left(1 + \omega^2 \tau^2 \right)} \right) + i \left( \frac{4 \pi \sigmaDC}{\omega \left( 1 + \omega^2 \tau^2 \right)} \right) \\
 &= \left(1 - \frac{4 \pi \sigmaDC  \tau}{1 + \omega^2 \tau^2 } \right) + i \left( \frac{4 \pi \sigmaDC}{\omega \left( 1 + \omega^2 \tau^2 \right)} \right) 
\end{align}

This lets us write down the explicit real and imaginary of the Drude dielectric function.

\subsection{Alternative forms of the Drude model}
We can also rewrite the dielectric constant very slightly in terms of the plasma frequency. In Gaussian units:
\begin{align}
	\epsilon &= 1 + i \frac{4 \pi \sigma}{\omega} \\
	&= 1 + i \frac{4 \pi}{\omega} \frac{n e^2 \tau}{m} \frac{1}{1 - i \omega \tau} \\
	&= 1 + i \frac{4 \pi}{\omega} \frac{n e^2 \tau}{m} \frac{1}{1 - i \omega \tau} \frac{i \nu}{i \nu} \\
	&= 1 - \frac{4 \pi}{\omega} \frac{n e^2 }{m} \frac{1}{i \nu + \omega }
\end{align}
With $\omega_p^2 = \frac{4 \pi n e^2}{m}$ in Gaussian units, this becomes 
\begin{align}
	\epsilon = 1 - \frac{\omega_p^2}{\omega ( \omega + i \nu )}
\end{align}
We'll see this again later.

\subsection{Derivations for Drude model}

\subsubsection{DC Conductivity}
We can start unit-system independently, with the expression 
\begin{equation}
	{\vec{j}} = \sigma \vec{E}. \label{eq:DrudeTheory:ConductivityDef}
\end{equation}
We can also relate our current to our average electron velocity: $\vec{j} = n e \vec{v}$. Imagine at time $t = 0$ our electron undergoes a Drude collision, and emerges with $\vec{v}_{t = 0} = \vec{v_0}$. After a time $t$, the electron will accelerate with acceleration $-\frac{e \vec{E}}{m}$ (which fortunately remains unit independent). Because it will only accelerate for a time $\tau$ on average before a collision, it will end up with velocity $\vec{v} = -\frac{e \vec{E}}{m}\tau + \vec{v_0}$. The average velocity, and current, will be 
\begin{align}
	\ev{\vec{v}} &= - \frac{e \vec{E}}{m}\tau + \ev{\vec{v_0}} \\
	&= - \frac{e \vec{E}}{m}\tau \\
	\frac{\vec{j}}{ne} &= - \frac{e \vec{E}}{m}\tau \\
	\vec{j} &= - \frac{n e^2 \tau}{m} \vec{E}.
\end{align}
This of course gives us, unit-independently, our DC conductivity $\sigmaDC = \frac{n e^2 \tau}{m}$. 	

\subsubsection{AC Conductivity} 

The AC conductivity is also simple, but we want to be a bit more formal about it. We can write out the contributions to velocity in terms of probabilities. The velocity at a time $\dd{t}$ will have probability $\flatfrac{\dd{t}}{\tau}$ of being $0$, and will otherwise be the original velocity minus $a \dd{t}$:
\begin{align}
	\vec{v}(\dd{t}) &= \left(1 - \frac{\dd{t}}{\tau}\right) \left(\vec{v_0} - \frac{e \vec{E}}{m} \dd{t} \right) \\
	&= \vec{v_0} - \frac{\dd{t}}{\tau}\vec{v_0} - \frac{e \vec{E}}{m} \dd{t},
\end{align}
where we've invoked our inalienable right as physicists to ignore all terms $\mathcal{O}(\dd{t}^2)$.
This reduces, using the definition of $\dd{\vec{v}} = \vec{v}(\dd{t}) - \vec{v_0}$, to
\begin{align}
	\dd{\vec{v}} &=  \frac{\dd{t}}{\tau} \vec{v} - \frac{e \vec{E}}{m} \dd{t} \\
	\dv{\vec{v}}{t} &= \frac{\vec{v}}{\tau} - \frac{e \vec{E}}{m}
\end{align}

We can quickly Fourier transform this, using $\dv{}{t} \rightarrow -i \omega$, and we get (after surreptitiously dropping some vector signs)
\begin{align}
	-i \omega v(\omega) &= - \frac{v(\omega)}{\tau} - \frac{e E(\omega)}{m} \\
	v(\omega) &= \frac{e E(\omega)}{m \left(\frac{1}{\tau} - i\omega \right)} \\
	j(\omega) &= \frac{n e^2 E(\omega)}{m \left(\frac{1}{\tau} - i\omega \right)} \\
	&= \frac{n e^2 \tau E(\omega)}{m \left(1 - i\omega \tau \right)}, 
\end{align}
which gives us our AC conductivity in equation \eqref{eq:DrudeTheory:SigmaAC}.

Now for our dielectric constant, we have to find some other defining relation on par with \eqref{eq:DrudeTheory:ConductivityDef}.

\end{document}
