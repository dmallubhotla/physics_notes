\documentclass[../../main.tex]{subfiles}
\newcommand{\sigmaDC}{\sigma_{\textrm{DC}}}
\newcommand{\sigmaAC}{\sigma_{\textrm{AC}}}


\begin{document}

% !TeX spellcheck = en_GB
\section{Drude model $\sigma$ and $\epsilon$}

The assumptions of the Drude model are simple: we have interaction-free electrons that occasionally undergo some scattering process during a time $\dd{t}$ with probability $\frac{\dd{t}}{\tau}$, where $\tau$ is some phenomenological parameter. This scattering will randomise electron momentum.

Our ultimate goal will be to find the conductivity $\sigma$ and the dielectric constant $\epsilon$ in the Drude model, with Drude relaxation time $\tau$, electron density $n$ and electron mass $m$. We'll find

\begin{align}
	\sigmaDC &= \frac{n e^2 \tau}{m} \\
	\sigmaAC &= \frac{n e^2 \tau}{m} \frac{1}{1 - i \omega \tau} && \text{For SI and Gaussian}
\end{align}

\begin{subequations}
	\begin{align}
		\epsilon_r &= 1 + i \frac{4 \pi \sigma}{\omega} && \text{Gaussian} \\
		\epsilon_r &= 1 + i \frac{\sigma}{\omega \epsilon_0} && \text{SI} 
	\end{align}
\end{subequations}

Our dielectric constant can be rewritten to plug in for $\sigma$, giving us
\begin{align}
	\epsilon &= 1 + i \frac{4 \pi \sigma}{\omega} \\
	&= 1 + i \frac{4 \pi \sigmaDC}{\omega} \frac{1}{1 - i \omega \tau} \\
 &= 1 + i \frac{4 \pi \sigmaDC}{\omega} \frac{1}{1 - i \omega \tau} \frac{1 + i \omega\tau}{1 + i \omega \tau} \\
 &= 1 + i \frac{4 \pi \sigmaDC}{\omega} \frac{1 + i \omega \tau}{1 + \omega^2 \tau^2} \\
 &= \left(1 - \frac{4 \pi \sigmaDC \omega \tau}{\omega\left(1 + \omega^2 \tau^2 \right)} \right) + i \left( \frac{4 \pi \sigmaDC}{\omega \left( 1 + \omega^2 \tau^2 \right)} \right) \\
 &= \left(1 - \frac{4 \pi \sigmaDC  \tau}{1 + \omega^2 \tau^2 } \right) + i \left( \frac{4 \pi \sigmaDC}{\omega \left( 1 + \omega^2 \tau^2 \right)} \right) 
\end{align}

This lets us write down the explicit real and imaginary of the Drude dielectric function.

\subsection{Limiting forms of the Drude model}
We can look at the large and small $\omega$ limits for the Drude dielectric function.

\subsection{Derivations for Drude model}

\subsubsection{Conductivity}
We can start unit-system independently, with the expression $\ev{\vec{j}} = \sigma \vec{E}$. We can also relate our current to our average electron velocity: $\vec{j} = n e \vec{v}$. Imagine at time $t = 0$ our electron undergoes a Drude collision, and emerges with $\vec{v}_{t = 0} = \vec{v_0}$. After a time $t$, the electron will accelerate with acceleration $-\frac{e \vec{E}}{m}$ (which fortunately remains unit independent). Because it will only accelerate for a time $\tau$ on average before a collision, it will end up with velocity $\vec{v} = -\frac{e \vec{E}}{m}\tau + \vec{v_0}$. The average velocity, and current, will be 
\begin{align}
	\ev{\vec{v}} &= - \frac{e \vec{E}}{m}\tau + \ev{\vec{v_0}} \\
	&= - \frac{e \vec{E}}{m}\tau \\
	\frac{\vec{j}}{ne} &= - \frac{e \vec{E}}{m}\tau \\
	\vec{j} &= - \frac{n e^2 \tau}{m} \vec{E}.
\end{align}
This of course gives us, unit-independently, our DC conductivity $\sigmaDC = \frac{n e^2 \tau}{m}$

\end{document}
