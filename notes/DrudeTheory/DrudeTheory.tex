\documentclass[../../main.tex]{subfiles}
\newcommand{\sigmaDC}{\sigma_{\textrm{DC}}}
\newcommand{\sigmaAC}{\sigma_{\textrm{AC}}}


\begin{document}

% !TeX spellcheck = en_GB
\section{Drude model $\sigma$ and $\epsilon$}

Our ultimate goal will be to find the conductivity $\sigma$ and the dielectric constant $\epsilon$ in the Drude model, with Drude relaxation time $\tau$, electron density $n$ and electron mass $m$. We'll find

\begin{gather}
	\sigmaDC = \frac{n e^2 \tau}{m} \\
	\sigmaAC = \frac{n e^2 \tau}{m} \frac{1}{1 - i \omega \tau} \\
	\epsilon = 1 + i \frac{4 \pi \sigma}{\omega}
\end{gather}

Our dielectric constant can be rewritten to plug in for $\sigma$, giving us
\begin{align}
	\epsilon &= 1 + i \frac{4 \pi \sigma}{\omega} \\
	&= 1 + i \frac{4 \pi \sigmaDC}{\omega} \frac{1}{1 - i \omega \tau} \\
 &= 1 + i \frac{4 \pi \sigmaDC}{\omega} \frac{1}{1 - i \omega \tau} \frac{1 + i \omega\tau}{1 + i \omega \tau} \\
 &= 1 + i \frac{4 \pi \sigmaDC}{\omega} \frac{1 + i \omega \tau}{1 + \omega^2 \tau^2} \\
 &= \left(1 - \frac{4 \pi \sigmaDC \omega \tau}{\omega\left(1 + \omega^2 \tau^2 \right)} \right) + i \left( \frac{4 \pi \sigmaDC}{\omega \left( 1 + \omega^2 \tau^2 \right)} \right) \\
 &= \left(1 - \frac{4 \pi \sigmaDC  \tau}{1 + \omega^2 \tau^2 } \right) + i \left( \frac{4 \pi \sigmaDC}{\omega \left( 1 + \omega^2 \tau^2 \right)} \right) 
\end{align}

This lets us write down the explicit real and imaginary of the Drude dielectric function.

\subsection{Derivations for Drude model}

\end{document}
