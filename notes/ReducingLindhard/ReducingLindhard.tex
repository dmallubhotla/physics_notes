\documentclass[../../main.tex]{subfiles}
\newcommand{\vf}{v_F}
\allowdisplaybreaks

\begin{document}

% !TeX spellcheck = en_GB
\section{Reducing Lindhard to Drude}
\subsection{The longitudinal case}
We want to see how we can reduce the longitudinal Lindhard dielectric function
\begin{equation} \label{eq:ReducingLindhard:LindhardEps} 
 \epsilon_l = 1 + \frac{3 \omega_p^2}{k^2 v_F^2} \frac{\left(\omega + i \nu \right) f_l(\flatfrac{(\omega + i \nu)}{k v_f})}{\omega + i \nu f_l(\flatfrac{(\omega + i \nu)}{k v_f})},
\end{equation}
where 
\begin{equation}
	f_l(x) = 1 - \frac{x}{2} \ln\frac{x + 1}{x - 1}.
\end{equation}

We can reduce things in the $k \rightarrow 0$ limit. The first half of \eqref{eq:ReducingLindhard:LindhardEps} has the simple $\frac{1}{k^2}$ dependence, so we can look at how the rest of it behaves to start with. 

\subsection{f}

\begin{align}
		f_l(\flatfrac{(\omega + i \nu)}{k v_F}) = 1 - \frac{\flatfrac{(\omega + i \nu)}{k v_F}}{2} \ln\frac{\flatfrac{(\omega + i \nu)}{k v_F} + 1}{\flatfrac{(\omega + i \nu)}{k v_F} - 1}
\end{align}

Defining $\eta = \omega + i \nu$:

\begin{align}
	f_l(\flatfrac{(\omega + i \nu)}{k v_f}) &= 1 - \frac{\flatfrac{(\omega + i \nu)}{k v_f}}{2} \ln\frac{\flatfrac{(\omega + i \nu)}{k v_f} + 1}{\flatfrac{(\omega + i \nu)}{k v_f} - 1} \\
	f_l(\flatfrac{\eta}{k \vf}) &= 1 - \frac{\eta}{2 k \vf} \ln\frac{\eta + k \vf}{\eta - k\vf} \\
	f_l &=  1 - \frac{\eta}{2 \vf} \frac{\ln\frac{\eta + k \vf}{\eta - k\vf}}{k} \\ 
	\lim_{k \to 0} f_l &= 1 - \frac{\eta}{2 \vf} \dv{\ln\frac{\eta + k \vf}{\eta - k\vf}}{k} \\
	&= 1 - \frac{\eta}{2 \vf} \frac{\eta - k \vf}{\eta + k\vf} \frac{\vf \left(\eta - k\vf \right)+ \vf \left(\eta + k \vf \right)}{\left( \eta - k \vf \right)^2} \\
	&= 1 - \frac{\eta}{2 } \frac{\eta - k \vf}{\eta + k\vf} \frac{\left(\eta - k\vf \right)+  \left(\eta + k \vf \right)}{\left( \eta - k \vf \right)^2} \\
	&= 1 - \frac{\eta}{2 } \frac{1}{\eta + k\vf} \frac{2 \eta}{\eta - k \vf } \\
	&= 1 - \frac{\eta^2}{\eta^2 - k^2 \vf^2} \\
	&= \frac{-k^2 \vf^2}{\eta^2 - k^2 \vf^2} \\
	\lim_{k \to 0} f_l &= 0
\end{align}
Note that this goes to $0$ for $k \to 0$.

\subsection{Series expansion of f}
The previous section gives the limit, but having the actual series expansion is probably more valuable. Again, with $\eta = \omega + i \nu$,
\begin{align}
	f_l(\flatfrac{(\omega + i \nu)}{k v_f}) &= 1 - \frac{\flatfrac{(\omega + i \nu)}{k v_f}}{2} \ln\frac{\flatfrac{(\omega + i \nu)}{k v_f} + 1}{\flatfrac{(\omega + i \nu)}{k v_f} - 1} \\
	f_l(\flatfrac{\eta}{k \vf}) &= 1 - \frac{\eta}{2 \vf k} \ln\frac{\eta + k \vf}{\eta - k\vf} \label{eq:ReducingLindhard:simplef}
\end{align}
We want to expand up to $k^2$ overall, to cancel out the $k^2$ in the denominator of \eqref{eq:ReducingLindhard:LindhardEps}. We're looking at the function. 
\begin{equation}
	\frac{\ln(\eta + k \vf)}{k} - \frac{\ln(\eta - k \vf)}{k} 
\end{equation}

Generally, the derivatives of $\frac{g(a \pm x)}{x}$ are
\begin{align}
	\left(\frac{g(x)}{x} \right)' &= \frac{\pm g'}{x} - \frac{g}{x^2} \\
	\left(\frac{g(x)}{x} \right)'' &= \frac{g''}{x} - \frac{\pm 2g'}{x^2}+ \frac{2 g}{x^3} \\
	\left(\frac{g(x)}{x} \right)''' &= \frac{\pm g'''}{x} - \frac{  3g''}{x^2}+ \frac{\pm 6 g'}{x^3} - \frac{6 g} {x^4}
\end{align}

When we take the difference, we see that we'll only end up keeping (and doubling) the terms of odd derivatives. Thus, up to this order, the series for $\frac{g(a + x) - g(a - x)}{x}$ will look like:
\begin{align}
	\frac12 \frac{g(a + x) - g(a - x)}{x} &= x \frac{g'}{x} - \frac12 x^2 \frac{2 g'}{x^2} + \frac16 x^3 \left( \frac{g'''}{x} + \frac{6 g'}{x^3} \right) \\
	&= g' - g' + \frac16 x^2 g''' + g' \\
	\frac{g(a + x) - g(a - x)}{x} &= 2g' + \frac13 x^2 g''' + \mathcal{O}(x^4)
\end{align}
This type of result is to be expected: we are starting with an even function. For $g = \ln(\eta + k \vf)$, we have
\begin{align}
	g'(k = 0) = \frac{\vf}{\eta} \\
	g'''(k = 0) = \frac{2 \vf^3}{\eta^3}
\end{align}
Plugging these into \eqref{eq:ReducingLindhard:simplef} gives us:
\begin{align}
	f_l(\flatfrac{\eta}{k \vf}) &= 1 - \frac{\eta}{2 \vf k} \ln\frac{\eta + k \vf}{\eta - k\vf} \\
	&= 1- \frac{\eta}{2\vf} \left( 2 \frac{\vf}{\eta} + \frac13 k^2 \frac{2 \vf^3}{\eta^3} \right) \\
	&= 1 - 1 - \frac13 k^2 \frac{\vf^2}{\eta^2}	\\
	&= - \frac{k^2 \vf^2}{3 \eta^2} \label{eq:ReducingLindhard:f}
\end{align}
This gives us a simple approximation for $f_l$ in the long wavelength limit.

\subsection{Back to dielectric}
\begin{equation} \label{eq:ReducingLindhard:EpsWithF} 
 \epsilon_l = 1 + \frac{3 \omega_p^2}{k^2 v_F^2} \frac{\left(\omega + i \nu \right) f_l(\flatfrac{(\omega + i \nu)}{k v_f})}{\omega + i \nu f_l(\flatfrac{(\omega + i \nu)}{k v_f})}
\end{equation}
In the denominator, we can note that $\omega$ should dominate $i \nu f$, because f goes to zero, so we can simplify that. 

\begin{equation}
	 \epsilon_l = 1 + \frac{3 \omega_p^2}{k^2 v_F^2} \frac{\left(\omega + i \nu \right) f_l(\flatfrac{(\omega + i \nu)}{k v_f})}{\omega}
\end{equation}
Using \eqref{eq:ReducingLindhard:f}, we get
\begin{align}
	\epsilon_l &= 1 + \frac{3 \omega_p^2}{k^2 v_F^2} \frac{\left(\omega + i \nu \right) f_l(\flatfrac{(\omega + i \nu)}{k v_f})}{\omega} \\
	&= 1 + \frac{3 \omega_p^2}{k^2 v_F^2} \frac{\left(\omega + i \nu \right) \frac{-k^2 \vf^2}{3 \eta^2}}{\omega} \\
	&= 1 - 3 \omega_p^2 \frac{\eta \frac{1}{ 3\eta^2}}{\omega} \\
	&= 1 - \frac{ \omega_p^2}{\omega \left( \omega + i \nu \right)}
\end{align}

This is the Drude limit, keeping in mind that $\omega_p^2 = \frac{4\pi n e^2}{m}$ in Gaussian units.

\subsection{Looking at transverse Lindhard form}
It's possible also to show that the transverse Lindhard dielectric function ends up also going to the Drude form in the $k \rightarrow 0$ limit. 

Writing out $\epsilon_t$:
\begin{equation} 
	e_t = 1 - \frac{\omega_p^2}{\omega \eta} \left[\frac32 \frac{\eta^2}{\vf^2 k^2} - \frac34 \frac{\eta}{\vf k} \left(\frac{\eta^2}{\vf^2 k^2} - 1 \right)\ln\frac{\eta + \vf k}{\eta - \vf k} \right]
\end{equation}

We can also simplify this slightly, to highlight the similarities between this and the longitudinal forms:
\begin{align}
	e_t &= 1 - \frac{3 \omega_p^2}{2\omega \eta} \left[ \frac{\eta^2}{\vf^2 k^2} - \frac12 \frac{\eta}{\vf k} \left(\frac{\eta^2}{\vf^2 k^2} - 1 \right)\ln\frac{\eta + \vf k}{\eta - \vf k} \right] \\
	&= 1 - \frac{3 \omega_p^2}{2\omega \eta} \left[ \frac{\eta^2}{\vf^2 k^2} - \frac12 \frac{\eta^2}{\vf^2 k^2} \left(\frac{\eta}{\vf k} - \frac{\vf k}{\eta} \right)\ln\frac{\eta + \vf k}{\eta - \vf k} \right] \\
	&= 1 - \frac{3 \omega_p^2}{2\omega \eta} \frac{\eta^2}{\vf^2 k^2} \left[ 1 - \frac12 \left(\frac{\eta}{\vf k} - \frac{\vf k}{\eta} \right)\ln\frac{\eta + \vf k}{\eta - \vf k} \right] \\
	&= 1 - \frac{3 \omega_p^2 \eta}{2\omega \vf^2 k^2} \left[ 1 - \frac12 \left(\frac{\eta}{\vf k} - \frac{\vf k}{\eta} \right)\ln\frac{\eta + \vf k}{\eta - \vf k} \right]\label{eq:ReducingLindhard:et}
\end{align}
There is an extra term within the brackets, as well as an extra factor of $-\frac12$ on the outside.

To find the $k \rightarrow 0$ limit here, we can do the same series expansion as we did for the longitudinal case. The relevant series to find is for the bracketed portion, which we can break into two parts:
\begin{align} \label{eq:ReducingLindhard:Simplerft}
	 1 - \frac12 \frac{\eta}{\vf k}\ln\frac{\eta + \vf k}{\eta - \vf k} - \frac12 \frac{\vf k}{\eta}\ln\frac{\eta + \vf k}{\eta - \vf k}
\end{align}

We have already done the expansion of the first two terms earlier, and we found that that should equal $- \frac{k^2 \vf^2}{3 \eta^2}$. We now need the series expansion of the third term. Like earlier, we can write out the derivatives of $k g(a \pm k)$, keeping in mind that the log will become a difference:
\begin{align}
	(k g)' &= g \pm k g' \\
	(k g)'' &= \pm 2 g' + k g''
\end{align}
After $k \rightarrow 0$ and the subtraction, the only term remaining will be the $2 \pm g'$ term. This will end up giving us
\begin{align}
	\frac12 \frac{\vf }{\eta} k \ln\frac{\eta + \vf k}{\eta - \vf k} &= \frac12 \frac{\vf}{\eta} \frac12 k^2 4  \frac{\vf}{\eta} \\
	&= \frac{k^2 \vf^2}{\eta^2} 
\end{align}

Adding this to \eqref{eq:ReducingLindhard:Simplerft}, we end up with 
\begin{align}
	1 - \frac12 \left(\frac{\eta}{\vf k} - \frac{\vf k}{\eta} \right)\ln\frac{\eta + \vf k}{\eta - \vf k} &= - \frac{k^2 \vf^2}{3 \eta^2} + \frac{k^2 \vf^2}{\eta^2} \\
	&= \frac23 \frac{k^2 \vf^2}{\eta^2}
\end{align}

Plugging this into \eqref{eq:ReducingLindhard:et} gives us
\begin{align}
	e_t &= 1 - \frac{3 \omega_p^2 \eta}{2\omega \vf^2 k^2} \left[ 1 - \frac12 \left(\frac{\eta}{\vf k} - \frac{\vf k}{\eta} \right)\ln\frac{\eta + \vf k}{\eta - \vf k} \right] \\
	&= 1 - \frac{3 \omega_p^2 \eta}{2\omega \vf^2 k^2} \frac23 \frac{k^2 \vf^2}{\eta^2} \\
	&= 1 - \frac{\omega_p^2}{\omega \eta}\\
	&= 1 - \frac{\omega_p^2}{\omega \left(\omega + i \nu \right) }
\end{align}
So to lowest order, both the longitudinal and transverse dielectric functions reduce to the Drude case in the $k \rightarrow 0$ limit. This makes intuitive sense: The Drude derivation assumes an isotropic $\vec{E}$. There should remain no distinction between transverse and longitudinal in this limit.

\section{Reducing Lindhard to Thomas-Fermi}

We can also try to recover the Thomas-Fermi dielectric function
\begin{equation}
	\epsilon_l = 1 + \frac{k_{TF}^2}{k^2}
\end{equation}

We'll see this in the static limit $\omega \rightarrow 0$.

\subsection{Longitudinal Lindhard form going to TF form}
\begin{equation} \label{eq:ReducingLindhard:LindhardEpsToTF}
 \epsilon_l = 1 + \frac{3 \omega_p^2}{k^2 v_F^2} \frac{\left(\omega + i \nu \right) f_l(\flatfrac{(\omega + i \nu)}{k v_f})}{\omega + i \nu f_l(\flatfrac{(\omega + i \nu)}{k v_f})},
\end{equation}

This limit is simpler to take:
\begin{align}
	 \epsilon_l &= 1 + \frac{3 \omega_p^2}{k^2 v_F^2} \frac{\left(\omega + i \nu \right) f_l(\flatfrac{(\omega + i \nu)}{k v_f})}{\omega + i \nu f_l(\flatfrac{(\omega + i \nu)}{k v_f})} \\
	 &= 1 + \frac{3 \omega_p^2}{k^2 v_F^2} \frac{i \nu  f_l(\flatfrac{(\omega + i \nu)}{k v_f})}{  i \nu f_l(\flatfrac{(\omega + i \nu)}{k v_f})} \\
	 &=  1 + \frac{3 \omega_p^2}{k^2 v_F^2}
\end{align}
Using the definition $k_{TF}^2 = \frac{3 \omega_p^2}{\vf^2}$, we rather easily get the Thomas-Fermi form.

It's worth noting that we get some disagreement between the Thomas-Fermi form and the Drude form around the region where both $k$ and $\omega$ go to zero.


\end{document}
