\documentclass[../../main.tex]{subfiles}
\newcommand{\vf}{v_F}
\allowdisplaybreaks

\begin{document}

% !TeX spellcheck = en_GB
\section{Reducing Lindhard to Drude}
We want to see how we can reduce the Lindhard dielectric function
\begin{equation} \label{eq:ReducingLindhard:LindhardEps} 
 \epsilon_l = 1 + \frac{3 \omega_p^2}{k^2 v_F^2} \frac{\left(\omega + i \nu \right) f_l(\flatfrac{(\omega + i \nu)}{k v_f})}{\omega + i \nu f_l(\flatfrac{(\omega + i \nu)}{k v_f})},
\end{equation}
where 
\begin{equation}
	f_l(x) = 1 - \frac{x}{2} \ln\frac{x + 1}{x - 1}
\end{equation}

We can reduce things in the $k \rightarrow 0$ limit. The first half of \eqref{eq:ReducingLindhard:LindhardEps} has the simple $\frac{1}{k^2}$ dependence, so we can look at how the rest of it behaves to start with. 

\subsection{f}

\begin{align}
		f_l(\flatfrac{(\omega + i \nu)}{k v_F}) = 1 - \frac{\flatfrac{(\omega + i \nu)}{k v_F}}{2} \ln\frac{\flatfrac{(\omega + i \nu)}{k v_F} + 1}{\flatfrac{(\omega + i \nu)}{k v_F} - 1}
\end{align}

Defining $\eta = \omega + i \nu$:

\begin{align}
	f_l(\flatfrac{(\omega + i \nu)}{k v_f}) &= 1 - \frac{\flatfrac{(\omega + i \nu)}{k v_f}}{2} \ln\frac{\flatfrac{(\omega + i \nu)}{k v_f} + 1}{\flatfrac{(\omega + i \nu)}{k v_f} - 1} \\
	f_l(\flatfrac{\eta}{k \vf}) &= 1 - \frac{\eta}{2 k \vf} \ln\frac{\eta + k \vf}{\eta - k\vf} \\
	f_l &=  1 - \frac{\eta}{2 \vf} \frac{\ln\frac{\eta + k \vf}{\eta - k\vf}}{k} \\ 
	\lim_{k \to 0} f_l &= 1 - \frac{\eta}{2 \vf} \dv{\ln\frac{\eta + k \vf}{\eta - k\vf}}{k} \\
	&= 1 - \frac{\eta}{2 \vf} \frac{\eta - k \vf}{\eta + k\vf} \frac{\vf \left(\eta - k\vf \right)+ \vf \left(\eta + k \vf \right)}{\left( \eta - k \vf \right)^2} \\
	&= 1 - \frac{\eta}{2 } \frac{\eta - k \vf}{\eta + k\vf} \frac{\left(\eta - k\vf \right)+  \left(\eta + k \vf \right)}{\left( \eta - k \vf \right)^2} \\
	&= 1 - \frac{\eta}{2 } \frac{1}{\eta + k\vf} \frac{2 \eta}{\eta - k \vf } \\
	&= 1 - \frac{\eta^2}{\eta^2 - k^2 \vf^2} \\
	&= \frac{-k^2 \vf^2}{\eta^2 - k^2 \vf^2} \\
	\lim_{k \to 0} f_l &= 0
\end{align}
Note that this goes to $0$ for $k \to 0$.

\subsection{Series expansion of f}
The previous section gives the limit, but having the actual series expansion is probably more valuable. Again, with $\eta = \omega + i \nu$,
\begin{align}
	f_l(\flatfrac{(\omega + i \nu)}{k v_f}) &= 1 - \frac{\flatfrac{(\omega + i \nu)}{k v_f}}{2} \ln\frac{\flatfrac{(\omega + i \nu)}{k v_f} + 1}{\flatfrac{(\omega + i \nu)}{k v_f} - 1} \\
	f_l(\flatfrac{\eta}{k \vf}) &= 1 - \frac{\eta}{2 \vf k} \ln\frac{\eta + k \vf}{\eta - k\vf} \label{eq:ReducingLindhard:simplef}
\end{align}
We want to expand up to $k^2$ overall, to cancel out the $k^2$ in the denominator of \eqref{eq:ReducingLindhard:LindhardEps}. We're looking at the function. 
\begin{equation}
	\frac{\ln(\eta + k \vf)}{k} - \frac{\ln(\eta - k \vf)}{k} 
\end{equation}

Generally, the derivatives of $\frac{g(a \pm x)}{x}$ are
\begin{align}
	\left(\frac{g(x)}{x} \right)' &= \frac{\pm g'}{x} - \frac{g}{x^2} \\
	\left(\frac{g(x)}{x} \right)'' &= \frac{g''}{x} - \frac{\pm 2g'}{x^2}+ \frac{2 g}{x^3} \\
	\left(\frac{g(x)}{x} \right)''' &= \frac{\pm g'''}{x} - \frac{  3g''}{x^2}+ \frac{\pm 6 g'}{x^3} - \frac{6 g} {x^4}
\end{align}

When we take the difference, we see that we'll only end up keeping (and doubling) the terms of odd derivatives. Thus, up to this order, the series for $\frac{g(a + x) - g(a - x)}{x}$ will look like:
\begin{align}
	\frac12 \frac{g(a + x) - g(a - x)}{x} &= x \frac{g'}{x} - \frac12 x^2 \frac{2 g'}{x^2} + \frac16 x^3 \left( \frac{g'''}{x} + \frac{6 g'}{x^3} \right) \\
	&= g' - g' + \frac16 x^2 g''' + g' \\
	\frac{g(a + x) - g(a - x)}{x} &= 2g' + \frac13 x^2 g''' + \mathcal{O}(x^4)
\end{align}
This type of result is to be expected: we are starting with an even function. For $g = \ln(\eta + k \vf)$, we have
\begin{align}
	g'(k = 0) = \frac{\vf}{\eta} \\
	g'''(k = 0) = \frac{2 \vf^3}{\eta^3}
\end{align}
Plugging these into \eqref{eq:ReducingLindhard:simplef} gives us:
\begin{align}
	f_l(\flatfrac{\eta}{k \vf}) &= 1 - \frac{\eta}{2 \vf k} \ln\frac{\eta + k \vf}{\eta - k\vf} \\
	&= 1- \frac{\eta}{2\vf} \left( 2 \frac{\vf}{\eta} + \frac13 k^2 \frac{2 \vf^3}{\eta^3} \right) \\
	&= 1 - 1 - \frac13 k^2 \frac{\vf^2}{\eta^2}	\\
	&= - \frac{k^2 \vf^2}{3 \eta^2} \label{eq:ReducingLindhard:f}
\end{align}
This gives us a simple approximation for $f_l$ in the long wavelength limit.

\subsection{Back to dielectric}
\begin{equation} \label{eq:ReducingLindhard:EpsWithF} 
 \epsilon_l = 1 + \frac{3 \omega_p^2}{k^2 v_F^2} \frac{\left(\omega + i \nu \right) f_l(\flatfrac{(\omega + i \nu)}{k v_f})}{\omega + i \nu f_l(\flatfrac{(\omega + i \nu)}{k v_f})}
\end{equation}
In the denominator, we can note that $\omega$ should dominate $i \nu f$, because f goes to zero, so we can simplify that. 

\begin{equation}
	 \epsilon_l = 1 + \frac{3 \omega_p^2}{k^2 v_F^2} \frac{\left(\omega + i \nu \right) f_l(\flatfrac{(\omega + i \nu)}{k v_f})}{\omega}
\end{equation}
Using \eqref{eq:ReducingLindhard:f}, we get
\begin{align}
	\epsilon_l &= 1 + \frac{3 \omega_p^2}{k^2 v_F^2} \frac{\left(\omega + i \nu \right) f_l(\flatfrac{(\omega + i \nu)}{k v_f})}{\omega} \\
	&= 1 + \frac{3 \omega_p^2}{k^2 v_F^2} \frac{\left(\omega + i \nu \right) \frac{-k^2 \vf^2}{3 \eta^2}}{\omega} \\
	&= 1 - 3 \omega_p^2 \frac{\eta \frac{1}{ 3\eta^2}}{\omega} \\
	&= 1 - \frac{ \omega_p^2}{\omega \left( \omega + i \nu \right)}
\end{align}

This is the Drude limit.

\end{document}
