\documentclass[../../main.tex]{subfiles}

\begin{document}

\newcommand{\frk}{k}
\newcommand{\frargs}{(\vec{\frk}, \omega)}

% !TeX spellcheck = en_GB
\section{Summary of Lindhard's derivation}

For Lindhard's derivation, we can start with two variations on the Maxwell's equations:
\begin{equation}
	\left(\frk^2 - \frac{\omega^2}{c^2} \epsilon^{tr}\frargs\right) \vec{A}^{tr}\frargs = \frac{4\pi}{c} \vec{j}^{tr}_{f}\frargs,
\end{equation}
\begin{equation}
	\epsilon^{lo}\frargs \frk^2 V\frargs = 4 \pi \rho_f\frargs.
\end{equation}
Here, $\rho_f$ and $j_f$ are the free charge density and current. The longitudinal and transverse dielectric functions, $\epsilon^{lo}$ and $\epsilon^{tr}$, contain the same information as the traditional $\epsilon$ and $\mu$, and are related by
\begin{gather}
	\epsilon\frargs = \epsilon^{lo}\frargs \\
	\frk^2 \left(1 - \frac{1}{\mu\frargs}\right) = \frac{\omega^2}{c^2} \left(\epsilon^{tr}\frargs - \epsilon^{lo}\frargs\right)
\end{gather}
Lindhard starts by looking at the Boltzmann equation:
\begin{equation} \label{eq:SummaryOfLindhardDerivation:BoltzmannEquation}
	\pdv{f}{t} + \vec{F} \cdot \grad_{p} f + \vec{v} \cdot \grad_{r} f = -\frac{f - f_0}{\tau}
\end{equation}
where $f(r, p, t)$ is the electron distribution function, $\vec{F}$ is the external force and $\tau$ is the relaxation time.\todo{Include note about the regimes where the Boltzmann equation holds. This could be where errors creep in.} We can define $f(r, p, t) = f_0(r, p) + f_1(r, p, t)$, where $f_0$ represents an equilibrium, time-independent electron distribution. Then, simplifying gives us:
\begin{align}
	-\frac{f - f_0}{\tau} &= \pdv{f}{t} + \vec{F} \cdot \grad_{p} f + \vec{v} \cdot \grad_{r} f \\
	-\frac{f_1}{\tau} &= \pdv{f_1}{t} + \vec{F} \cdot \grad_{p} \left(f_0 + f_1\right) + \vec{v} \cdot \grad_{r} \left(f_0 + f_1\right) \label{eq:SummaryOfLindhardDerivation:BoltzmannIntermediateStep}
\end{align}
Because $f_0$ is an equilibrium solution, we know that it must separately satisfy a time-independent Boltzmann equation:
\begin{align}
	0 &= \vec{F} \cdot \grad_{p} f_0 + \vec{v}\cdot \grad_{r} f_0,
\end{align}
which means we can reduce \eqref{eq:SummaryOfLindhardDerivation:BoltzmannIntermediateStep} to
\begin{align}
	-\frac{f_1}{\tau} &= \pdv{f_1}{t} + \vec{F} \cdot \grad_{p} \left(f_0 + f_1\right) + \vec{v} \cdot \grad_{r} \left(f_0 + f_1\right) \\
	-\frac{f_1}{\tau} &= 
\end{align}
\end{document}
