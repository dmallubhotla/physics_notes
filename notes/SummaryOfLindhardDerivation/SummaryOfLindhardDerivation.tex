\documentclass[../../main.tex]{subfiles}

\begin{document}

\newcommand{\frk}{k}
\newcommand{\frargs}{(\vec{\frk}, \omega)}

% !TeX spellcheck = en_GB
\section{Summary of Lindhard's derivation}

For Lindhard's derivation, we can start with two variations on the Maxwell's equations:
\begin{equation}
	\left(\frk^2 - \frac{\omega^2}{c^2} \epsilon^{tr}\frargs\right) \vec{A}^{tr}\frargs = \frac{4\pi}{c} \vec{j}^{tr}_{f}\frargs,
\end{equation}
\begin{equation}
	\epsilon^{lo}\frargs \frk^2 V\frargs = 4 \pi \rho_f\frargs.
\end{equation}
Here, $\rho_f$ and $j_f$ are the free charge density and current. The longitudinal and transverse dielectric functions, $\epsilon^{lo}$ and $\epsilon^{tr}$, contain the same information as the traditional $\epsilon$ and $\mu$, and are related by
\begin{gather}
	\epsilon\frargs = \epsilon^{lo}\frargs \\
	\frk^2 \left(1 - \frac{1}{\mu\frargs}\right) = \frac{\omega^2}{c^2} \left(\epsilon^{tr}\frargs - \epsilon^{lo}\frargs\right)
\end{gather}


\end{document}
